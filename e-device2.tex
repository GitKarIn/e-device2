% Options for packages loaded elsewhere
\PassOptionsToPackage{unicode}{hyperref}
\PassOptionsToPackage{hyphens}{url}
%
\documentclass[
]{book}
\usepackage{amsmath,amssymb}
\usepackage{iftex}
\ifPDFTeX
  \usepackage[T1]{fontenc}
  \usepackage[utf8]{inputenc}
  \usepackage{textcomp} % provide euro and other symbols
\else % if luatex or xetex
  \usepackage{unicode-math} % this also loads fontspec
  \defaultfontfeatures{Scale=MatchLowercase}
  \defaultfontfeatures[\rmfamily]{Ligatures=TeX,Scale=1}
\fi
\usepackage{lmodern}
\ifPDFTeX\else
  % xetex/luatex font selection
\fi
% Use upquote if available, for straight quotes in verbatim environments
\IfFileExists{upquote.sty}{\usepackage{upquote}}{}
\IfFileExists{microtype.sty}{% use microtype if available
  \usepackage[]{microtype}
  \UseMicrotypeSet[protrusion]{basicmath} % disable protrusion for tt fonts
}{}
\makeatletter
\@ifundefined{KOMAClassName}{% if non-KOMA class
  \IfFileExists{parskip.sty}{%
    \usepackage{parskip}
  }{% else
    \setlength{\parindent}{0pt}
    \setlength{\parskip}{6pt plus 2pt minus 1pt}}
}{% if KOMA class
  \KOMAoptions{parskip=half}}
\makeatother
\usepackage{xcolor}
\usepackage{color}
\usepackage{fancyvrb}
\newcommand{\VerbBar}{|}
\newcommand{\VERB}{\Verb[commandchars=\\\{\}]}
\DefineVerbatimEnvironment{Highlighting}{Verbatim}{commandchars=\\\{\}}
% Add ',fontsize=\small' for more characters per line
\usepackage{framed}
\definecolor{shadecolor}{RGB}{248,248,248}
\newenvironment{Shaded}{\begin{snugshade}}{\end{snugshade}}
\newcommand{\AlertTok}[1]{\textcolor[rgb]{0.94,0.16,0.16}{#1}}
\newcommand{\AnnotationTok}[1]{\textcolor[rgb]{0.56,0.35,0.01}{\textbf{\textit{#1}}}}
\newcommand{\AttributeTok}[1]{\textcolor[rgb]{0.13,0.29,0.53}{#1}}
\newcommand{\BaseNTok}[1]{\textcolor[rgb]{0.00,0.00,0.81}{#1}}
\newcommand{\BuiltInTok}[1]{#1}
\newcommand{\CharTok}[1]{\textcolor[rgb]{0.31,0.60,0.02}{#1}}
\newcommand{\CommentTok}[1]{\textcolor[rgb]{0.56,0.35,0.01}{\textit{#1}}}
\newcommand{\CommentVarTok}[1]{\textcolor[rgb]{0.56,0.35,0.01}{\textbf{\textit{#1}}}}
\newcommand{\ConstantTok}[1]{\textcolor[rgb]{0.56,0.35,0.01}{#1}}
\newcommand{\ControlFlowTok}[1]{\textcolor[rgb]{0.13,0.29,0.53}{\textbf{#1}}}
\newcommand{\DataTypeTok}[1]{\textcolor[rgb]{0.13,0.29,0.53}{#1}}
\newcommand{\DecValTok}[1]{\textcolor[rgb]{0.00,0.00,0.81}{#1}}
\newcommand{\DocumentationTok}[1]{\textcolor[rgb]{0.56,0.35,0.01}{\textbf{\textit{#1}}}}
\newcommand{\ErrorTok}[1]{\textcolor[rgb]{0.64,0.00,0.00}{\textbf{#1}}}
\newcommand{\ExtensionTok}[1]{#1}
\newcommand{\FloatTok}[1]{\textcolor[rgb]{0.00,0.00,0.81}{#1}}
\newcommand{\FunctionTok}[1]{\textcolor[rgb]{0.13,0.29,0.53}{\textbf{#1}}}
\newcommand{\ImportTok}[1]{#1}
\newcommand{\InformationTok}[1]{\textcolor[rgb]{0.56,0.35,0.01}{\textbf{\textit{#1}}}}
\newcommand{\KeywordTok}[1]{\textcolor[rgb]{0.13,0.29,0.53}{\textbf{#1}}}
\newcommand{\NormalTok}[1]{#1}
\newcommand{\OperatorTok}[1]{\textcolor[rgb]{0.81,0.36,0.00}{\textbf{#1}}}
\newcommand{\OtherTok}[1]{\textcolor[rgb]{0.56,0.35,0.01}{#1}}
\newcommand{\PreprocessorTok}[1]{\textcolor[rgb]{0.56,0.35,0.01}{\textit{#1}}}
\newcommand{\RegionMarkerTok}[1]{#1}
\newcommand{\SpecialCharTok}[1]{\textcolor[rgb]{0.81,0.36,0.00}{\textbf{#1}}}
\newcommand{\SpecialStringTok}[1]{\textcolor[rgb]{0.31,0.60,0.02}{#1}}
\newcommand{\StringTok}[1]{\textcolor[rgb]{0.31,0.60,0.02}{#1}}
\newcommand{\VariableTok}[1]{\textcolor[rgb]{0.00,0.00,0.00}{#1}}
\newcommand{\VerbatimStringTok}[1]{\textcolor[rgb]{0.31,0.60,0.02}{#1}}
\newcommand{\WarningTok}[1]{\textcolor[rgb]{0.56,0.35,0.01}{\textbf{\textit{#1}}}}
\usepackage{longtable,booktabs,array}
\usepackage{calc} % for calculating minipage widths
% Correct order of tables after \paragraph or \subparagraph
\usepackage{etoolbox}
\makeatletter
\patchcmd\longtable{\par}{\if@noskipsec\mbox{}\fi\par}{}{}
\makeatother
% Allow footnotes in longtable head/foot
\IfFileExists{footnotehyper.sty}{\usepackage{footnotehyper}}{\usepackage{footnote}}
\makesavenoteenv{longtable}
\usepackage{graphicx}
\makeatletter
\def\maxwidth{\ifdim\Gin@nat@width>\linewidth\linewidth\else\Gin@nat@width\fi}
\def\maxheight{\ifdim\Gin@nat@height>\textheight\textheight\else\Gin@nat@height\fi}
\makeatother
% Scale images if necessary, so that they will not overflow the page
% margins by default, and it is still possible to overwrite the defaults
% using explicit options in \includegraphics[width, height, ...]{}
\setkeys{Gin}{width=\maxwidth,height=\maxheight,keepaspectratio}
% Set default figure placement to htbp
\makeatletter
\def\fps@figure{htbp}
\makeatother
\setlength{\emergencystretch}{3em} % prevent overfull lines
\providecommand{\tightlist}{%
  \setlength{\itemsep}{0pt}\setlength{\parskip}{0pt}}
\setcounter{secnumdepth}{5}
\usepackage{booktabs}
\usepackage{booktabs}
\usepackage{longtable}
\usepackage{array}
\usepackage{multirow}
\usepackage{wrapfig}
\usepackage{float}
\usepackage{colortbl}
\usepackage{pdflscape}
\usepackage{tabu}
\usepackage{threeparttable}
\usepackage{threeparttablex}
\usepackage[normalem]{ulem}
\usepackage{makecell}
\usepackage{xcolor}
\ifLuaTeX
  \usepackage{selnolig}  % disable illegal ligatures
\fi
\usepackage[]{natbib}
\bibliographystyle{apalike}
\IfFileExists{bookmark.sty}{\usepackage{bookmark}}{\usepackage{hyperref}}
\IfFileExists{xurl.sty}{\usepackage{xurl}}{} % add URL line breaks if available
\urlstyle{same}
\hypersetup{
  pdftitle={Electronic Data Entry Software Options for IEP Surveys},
  pdfauthor={IEP DUWG `e-Device' Sub-group led by Karrin Alstad},
  hidelinks,
  pdfcreator={LaTeX via pandoc}}

\title{Electronic Data Entry Software Options for IEP Surveys}
\author{IEP DUWG `e-Device' Sub-group led by Karrin Alstad}
\date{2023-05-04}

\usepackage{amsthm}
\newtheorem{theorem}{Theorem}[chapter]
\newtheorem{lemma}{Lemma}[chapter]
\newtheorem{corollary}{Corollary}[chapter]
\newtheorem{proposition}{Proposition}[chapter]
\newtheorem{conjecture}{Conjecture}[chapter]
\theoremstyle{definition}
\newtheorem{definition}{Definition}[chapter]
\theoremstyle{definition}
\newtheorem{example}{Example}[chapter]
\theoremstyle{definition}
\newtheorem{exercise}{Exercise}[chapter]
\theoremstyle{definition}
\newtheorem{hypothesis}{Hypothesis}[chapter]
\theoremstyle{remark}
\newtheorem*{remark}{Remark}
\newtheorem*{solution}{Solution}
\begin{document}
\maketitle

{
\setcounter{tocdepth}{1}
\tableofcontents
}
\hypertarget{about}{%
\chapter{About}\label{about}}

This is a \emph{sample} book written in \textbf{Markdown}. You can use anything that Pandoc's Markdown supports; for example, a math equation \(a^2 + b^2 = c^2\).

\hypertarget{usage}{%
\section{Usage}\label{usage}}

Each \textbf{bookdown} chapter is an .Rmd file, and each .Rmd file can contain one (and only one) chapter. A chapter \emph{must} start with a first-level heading: \texttt{\#\ A\ good\ chapter}, and can contain one (and only one) first-level heading.

Use second-level and higher headings within chapters like: \texttt{\#\#\ A\ short\ section} or \texttt{\#\#\#\ An\ even\ shorter\ section}.

The \texttt{index.Rmd} file is required, and is also your first book chapter. It will be the homepage when you render the book.

\hypertarget{render-book}{%
\section{Render book}\label{render-book}}

You can render the HTML version of this example book without changing anything:

\begin{enumerate}
\def\labelenumi{\arabic{enumi}.}
\item
  Find the \textbf{Build} pane in the RStudio IDE, and
\item
  Click on \textbf{Build Book}, then select your output format, or select ``All formats'' if you'd like to use multiple formats from the same book source files.
\end{enumerate}

Or build the book from the R console:

\begin{Shaded}
\begin{Highlighting}[]
\NormalTok{bookdown}\SpecialCharTok{::}\FunctionTok{render\_book}\NormalTok{()}
\end{Highlighting}
\end{Shaded}

To render this example to PDF as a \texttt{bookdown::pdf\_book}, you'll need to install XeLaTeX. You are recommended to install TinyTeX (which includes XeLaTeX): \url{https://yihui.org/tinytex/}.

\hypertarget{preview-book}{%
\section{Preview book}\label{preview-book}}

As you work, you may start a local server to live preview this HTML book. This preview will update as you edit the book when you save individual .Rmd files. You can start the server in a work session by using the RStudio add-in ``Preview book'', or from the R console:

\begin{Shaded}
\begin{Highlighting}[]
\NormalTok{bookdown}\SpecialCharTok{::}\FunctionTok{serve\_book}\NormalTok{()}
\end{Highlighting}
\end{Shaded}

\hypertarget{introduction}{%
\chapter{Introduction}\label{introduction}}

All chapters start with a first-level heading followed by your chapter title, like the line above. There should be only one first-level heading (\texttt{\#}) per .Rmd file.

\hypertarget{e-device-working-group-overview}{%
\section{E-device Working Group Overview}\label{e-device-working-group-overview}}

\hypertarget{goals}{%
\subsection{Goals}\label{goals}}

In general, the goal of the electronic data entry sub-group of the DUWG is to research field data entry software and hardware devices (``e-devices''), and to generate resources that support IEP survey leads in selecting and deploying digital data-entry procedures. This sub-group does not intend to suggest a single solution or software choice for all IEP surveys; rather, the sub-group aims to provide specific application reviews and testing, some methods development, and the start of an IEP e-device users network in effort to facilitate the independent decisions and potential transition of each IEP surveys to electronic data collection methods..

\hypertarget{approach}{%
\subsection{Approach}\label{approach}}

A general approach to the exploration of e-device applications was outlined and agreed on at the initial e-device meetings.

First, an e-device information gathering questionnaire (survey) would be distributed within IEP which specifically collects response data from: 1. Experienced e-device users, 2. Those who are currently researching e-device solutions for their survey applications; and 3. Those who have determined e-devices will not work for them.

Second, following leads from the responses to the survey, this distribution is expanded to include external associates (ICF, NEON, CDFW Marine). Follow-up interviews of the experienced e-device users and vendors are conducted, and demonstrations are arranged for the most promising e-device options.

Third, group members will potentially test specific e-devices apps by making use of free trial licenses, and these trials reported back to the group (SFBS \& Yolo By-pass were early volunteers for trial forms development exercises). Potentially other e-device methods will be researched and developed for demonstration purposes (e.g., collection and integration of external sensor data into e-device applications).

\hypertarget{scope-of-group-activities-and-intended-products}{%
\subsection{Scope of group activities and intended products}\label{scope-of-group-activities-and-intended-products}}

The initial e-device questionnaire was used to identify the most common e-device apps used within IEP (Table 1), the key questions/concerns about using e-devices (section below), as well as the key criteria that will be used to evaluate each software options explored by the working group (Tables 2-7). The main categories considering include options within the forms building tools, including QC related factors such as constrained choices and rules that guide subsequent fields. Other categories include IT security protocols, photo integration, cost of different product options, and factors related to the business model of each vendor, including the longevity of the company and the level/cost of customer/technical support.

\hypertarget{cross}{%
\chapter{E-device Survey}\label{cross}}

An Electronic Field Data-Entry Device (``e-Device'') Survey was distributed to IEP survey leads in July of 2022. The goals of this survey were: 1. To get an understanding of the current level of use of electronic field data entry devices among IEP Survey Staff; 2. To seek information from expert e-device users about both hardware and software considerations, and 3. To the learn specific roadblocks for those who are hesitant to consider shifting to electronic data entry methods.

Three groups were targeted for this survey:\\
1. Experienced e-device users (internal IEP and external agencies),\\
2. IEP Survey Staff who are researching e-devices for field application,\\
3. IEP Survey Staff who don't believe electronic field data collection will work for their application.

\hypertarget{survey-respondents}{%
\section{Survey Respondents}\label{survey-respondents}}

The IEP e-device survey was run for approximately a month and received 24 responses (Figure \ref{fig:survey}. A link to the original survey and to the compiled survey responses is included in the Appendix 2.

\begin{figure}

{\centering \includegraphics[width=0.8\linewidth]{02-survey_files/figure-latex/survey-1} 

}

\caption{The distribution of of e-device survey responders by agency association. Most of these respondents were CDFW staff, but USFWS and DWR responses were also represented.  Two external agencies (NEON and ICF) were specifically asked to participate after survey responses pointed to these expert resources.}\label{fig:survey}
\end{figure}

Among the 24 respondents, half of these were IEP associates and external contacts that are already using e-device applications for their survey data collections. Five (20\%) of the respondents were IEP associates who were currently seeking e-device solutions for their survey data collections. Two of the respondents indicated that they have already determined that electronic data entry would not work for their IEP survey application, and 5 respondents did not answer the question about their e-device use experience/status. An experienced e-device user resource list has been initiated, including contact names, software type, and type of survey application; this list will be expanded as possible (Appendix ).

\hypertarget{esri-survey123}{%
\chapter{ESRI Survey123}\label{esri-survey123}}

\hypertarget{survey123-overview}{%
\section{Survey123 Overview}\label{survey123-overview}}

From ESRI documentation: ArcGIS Survey123 is a complete, form-centric solution for creating, sharing, and analyzing surveys. Use it to create forms with skip logic, defaults, and support for multiple languages. Collect data using web or mobile devices, even when disconnected from the internet. Upload data securely, and analyze results on the web or in an ArcGIS app.
\url{https://doc.arcgis.com/en/survey123/reference/whatissurvey123.htm}

From G2 Business Software Review: Survey123 is included with ArcGIS, and provides powerful features to help you leverage the power of location to boost your productivity while capturing data and analyzing the results of your surveys. \url{https://www.g2.com/products/arcgis-survey123/reviews}

\hypertarget{survey123-forms-options}{%
\section{Survey123 Forms Options}\label{survey123-forms-options}}

Surveys123 Survey Forms are created and stored through the ESRI web interface (see Figure 1); access requires a current ESRI license. Survey123 forms can be downloaded to tablets, iPhones or iPads, and data collection can be made while the device is offline. Survey results are uploaded to cloud storage next time the device is on-line.

\begin{figure}
\includegraphics[width=0.9\linewidth]{figures/survey123/ESRIweb} \caption{Screen capture of the ESRI web interface for Survey123 and the option to Create New Survey.}\label{fig:ESRIweb}
\end{figure}

There are two main options for designing a Survey123 survey form: the Web Designer or Survey123 Connect tool. The Web Designer is a web-based menu-driven GUI that does not require learning any specific coding to set up a basic survey form. Survey questions and response types can be specified using a drag and drop tool.

\begin{figure}
\includegraphics[width=0.9\linewidth]{figures/survey123/ESRIweb2} \caption{Screen capture of the ESRI web interface for Survey123 highlighting the two main options for designing a Survey123 form: the web designer or the Survey123 Connect tool.}\label{fig:ESRIweb2}
\end{figure}

Survey123 Connect is an option for more advanced survey form design, such as a nested structure or calculated responses using user inputs. The Survey123 Connect approach requires defining the more advanced form properties within an `XLSForm spreadsheet' using the XLSForm coding language. ESRI documentation for both Web designer and Connect can be found at: \url{https://doc.arcgis.com/en/survey123/browser/create-surveys/createsurveys.htm}
XLSForm formatting language is described at: \url{https://xlsform.org/en/}.

\begin{figure}
\includegraphics[width=0.9\linewidth]{figures/survey123/survey123_xlsform} \caption{Screen capture of ESRI Survey123 Connect software demonstrating the use of the XLSForm spreadsheet-based coding language and the ability to directly edit the JavaScript code that is linked to XLSForm parameters.}\label{fig:xlsform}
\end{figure}

\begin{figure}
\includegraphics[width=0.9\linewidth]{figures/survey123/survey123_xlsform2} \caption{Screen capture of ArcGIS website description of the XLSForm features: https://gis.idaho.gov/wp-content/uploads/2021/03/ArcGIS-Apps-for-the-Field-State-of-ID.pdf.}\label{fig:xlsform2}
\end{figure}

\hypertarget{survey123-forms-criteria-table}{%
\section{Survey123 Forms Criteria Table}\label{survey123-forms-criteria-table}}

\hypertarget{footnotes-and-citations}{%
\chapter{Footnotes and citations}\label{footnotes-and-citations}}

\hypertarget{footnotes}{%
\section{Footnotes}\label{footnotes}}

Footnotes are put inside the square brackets after a caret \texttt{\^{}{[}{]}}. Like this one \footnote{This is a footnote.}.

\hypertarget{citations}{%
\section{Citations}\label{citations}}

Reference items in your bibliography file(s) using \texttt{@key}.

For example, we are using the \textbf{bookdown} package \citep{R-bookdown} (check out the last code chunk in index.Rmd to see how this citation key was added) in this sample book, which was built on top of R Markdown and \textbf{knitr} \citep{xie2015} (this citation was added manually in an external file book.bib).
Note that the \texttt{.bib} files need to be listed in the index.Rmd with the YAML \texttt{bibliography} key.

The \texttt{bs4\_book} theme makes footnotes appear inline when you click on them. In this example book, we added \texttt{csl:\ chicago-fullnote-bibliography.csl} to the \texttt{index.Rmd} YAML, and include the \texttt{.csl} file. To download a new style, we recommend: \url{https://www.zotero.org/styles/}

The RStudio Visual Markdown Editor can also make it easier to insert citations: \url{https://rstudio.github.io/visual-markdown-editing/\#/citations}

\hypertarget{blocks}{%
\chapter{Blocks}\label{blocks}}

\hypertarget{equations}{%
\section{Equations}\label{equations}}

Here is an equation.

\begin{equation} 
  f\left(k\right) = \binom{n}{k} p^k\left(1-p\right)^{n-k}
  \label{eq:binom}
\end{equation}

You may refer to using \texttt{\textbackslash{}@ref(eq:binom)}, like see Equation \eqref{eq:binom}.

\hypertarget{theorems-and-proofs}{%
\section{Theorems and proofs}\label{theorems-and-proofs}}

Labeled theorems can be referenced in text using \texttt{\textbackslash{}@ref(thm:tri)}, for example, check out this smart theorem \ref{thm:tri}.

\begin{theorem}
\protect\hypertarget{thm:tri}{}\label{thm:tri}For a right triangle, if \(c\) denotes the \emph{length} of the hypotenuse
and \(a\) and \(b\) denote the lengths of the \textbf{other} two sides, we have
\[a^2 + b^2 = c^2\]
\end{theorem}

Read more here \url{https://bookdown.org/yihui/bookdown/markdown-extensions-by-bookdown.html}.

\hypertarget{callout-blocks}{%
\section{Callout blocks}\label{callout-blocks}}

The \texttt{bs4\_book} theme also includes special callout blocks, like this \texttt{.rmdnote}.

You can use \textbf{markdown} inside a block.

\begin{Shaded}
\begin{Highlighting}[]
\FunctionTok{head}\NormalTok{(beaver1, }\AttributeTok{n =} \DecValTok{5}\NormalTok{)}
\CommentTok{\#\textgreater{}   day time  temp activ}
\CommentTok{\#\textgreater{} 1 346  840 36.33     0}
\CommentTok{\#\textgreater{} 2 346  850 36.34     0}
\CommentTok{\#\textgreater{} 3 346  900 36.35     0}
\CommentTok{\#\textgreater{} 4 346  910 36.42     0}
\CommentTok{\#\textgreater{} 5 346  920 36.55     0}
\end{Highlighting}
\end{Shaded}

It is up to the user to define the appearance of these blocks for LaTeX output.

You may also use: \texttt{.rmdcaution}, \texttt{.rmdimportant}, \texttt{.rmdtip}, or \texttt{.rmdwarning} as the block name.

The R Markdown Cookbook provides more help on how to use custom blocks to design your own callouts: \url{https://bookdown.org/yihui/rmarkdown-cookbook/custom-blocks.html}

\hypertarget{appendices}{%
\chapter{Appendices}\label{appendices}}

\hypertarget{e-device-survey}{%
\section{E-device Survey}\label{e-device-survey}}

\begin{table}

\caption{(\#tab:e_surv)Responses from Experience Users}
\centering
\begin{tabular}[t]{>{\raggedright\arraybackslash}p{4.5cm}|>{\raggedright\arraybackslash}p{4.5cm}|>{\raggedright\arraybackslash}p{4.5cm}|>{\raggedright\arraybackslash}p{4.5cm}|>{\raggedright\arraybackslash}p{4.5cm}}
\hline
Agency & field\_app & what\_hard & what\_soft & yr\_reasons\\
\hline
CDFW & 1. The Trimble GPS data dictionary
2. ESRI's "FieldMaps" app - tested with android and apple tablets and phones & Trimble GPS (not sure which unit) - cost about 10,000 dollars each; software licensing separate.
The FieldMaps app is free - can be used with any tablet or phone but works best with apple devices. We use it with a BadElf GNSS unit for GPS position information but accuracy is not great (cost - 700 dollars) & 1. GPS pathfinder for data dictionary for the Trimble GPS
\em{\textbf{2. ArcGIS online for the FieldMaps app.}} & \em{\textbf{1. The trimble positional accuracy is great but it is very pricey. 2. The FieldMaps is free (probably because UCD has a huge contract with ESRI) but the Bad Elf accuracy is no better than a phone.}} & \em{\textbf{NA}} & \em{\textbf{NA}} & \em{\textbf{NA}}\\
\hline
\em{\textbf{CDFW}} & \em{\textbf{Location, species, length, weight, water quality.}} & \em{\textbf{iPad, tablet.}} & \em{\textbf{Custom software developed by contractors or internal state IT deparment.}} & \em{\textbf{Purchase restrictions, ease of use, and formatting.}}\\
\hline
\em{\textbf{CDFW}} & \em{\textbf{Used iPad for electronic data entry}} & \em{\textbf{Not sure}} & \em{\textbf{Not sure}} & \em{\textbf{Not sure}}\\
\hline
\em{\textbf{USFWS}} & \em{\textbf{Salmon and Steelhead spawning ground surveys}} & \em{\textbf{Juniper Systems Mesa Tablet}} & \em{\textbf{ArcCollector}} & \em{\textbf{NA}}\\
\hline
USFWS & I am a data manager on the tributary monitoring team of the red bluff USFWS office. We collect all of our data digitally, using either field tablet, laptop or cell phone. 
1)spawning surveys
2)Rotary screw trap data
3)Habitat surveys & 1) Juniper mesa handheld tablet and juniper Mesa receiver, transitioning to ipad for handheld
2)Juniper mesa handheld tablet transitioning (back) to Panasonic Toughbook laptop
3) Same as 1 & 1) ESRI Collector
2) Access form
\em{\textbf{3) ESRI Collector}} & \em{\textbf{Our applications require waterproof devices. This shapes much of our hardware selection. Juniper Mesa and Juniper Mesa were selected for our purposes following some extended testing in 2017. We are transitioning away from the Juniper Mesa tablet as ESRI field maps in not being developed for windows OS (and USFWS does not support the other OS offered android). We are transitioning to IPADs as they are what USFWS supports and will work with our intended software.}} & \em{\textbf{NA}} & \em{\textbf{NA}} & \em{\textbf{NA}}\\
\hline
USGS & Water quality
Velocity
Discharge
Water Level
Field Notes & Laptops (usually Dell -- various models) & SVMAQ (USGS Site Visit)
\em{\textbf{Win River (Teledyne)}} & \em{\textbf{laptops are synchronized weekly with station meta data and any field software updates through the USGS network.  They are powerful enough to run the various software packages and are the systems that our team uses for office-based tasks as well.}} & \em{\textbf{NA}} & \em{\textbf{NA}} & \em{\textbf{NA}}\\
\hline
\em{\textbf{DWR}} & \em{\textbf{water quality and fish data (from beach seine, screw trap, fyke trap)}} & \em{\textbf{ipad - not sure what model}} & \em{\textbf{Survey123}} & \em{\textbf{ipads were already being used by others in our department and the software was free}}\\
\hline
UC & 1) geolocation of FAV and emergent vegetation patches and genera and associated characteristics (plant morphology and phenology, patch dimensions, percent cover, water quality).
2) UAV-mapping of FAV emergent veg and SAV - eDevices used for flight planning and flight control & 1). Trimble Geo7x Handheld Data recorder and GNSS receiver.
2.) A variety of android 4G tablets and phones (mostly samsung, but others as well). & 1.) Trimble commercial software - Devices runs windows mobile, with Trimble TerraSync for data collection. Trimble Pathfinder desktop software required for post-processing of files. Post-processed files compatible with any  OGC-standard GIS software.
2.) DJI flight planner, Drone Deploy, and Pix4D (all have various strengths and weaknesses in the field). & 1.) Ease of use, integrated camera and laser range finder, high accuracy GNSS location, and full-integration with GIS and compatibility with ESRI. Warning, these are pricey, but if you need cm-scale GNSS locations, these are some of the best for handheld devices. 
\em{\textbf{2.) We're still exploring best options. Usually start with manufacturer installs and recommendations and go from there. No strong opinions yet.}} & \em{\textbf{NA}} & \em{\textbf{NA}} & \em{\textbf{NA}} & \em{\textbf{NA}}\\
\hline
DWR & I piloted use of e-devices for fish and zooplankton surveys with the fish restoration program. Types of data:
1. Location of survey (latitude, logitude)
2. Water quality information
3. Trawl information (start time, stop time, gear used, etc)
4. Fish catch (lenghts, species) & iPads, don't remember the brand
Trimble field computers & Pendragon forms
\em{\textbf{Experimented with Survey 123 and ArcCollector}} & \em{\textbf{Pendragon forms allowed for more nested forms and flexibility than any of the other options. Ipads were the cheapest and most user-friendly option that came with weatherproff cases.}} & \em{\textbf{NA}} & \em{\textbf{NA}} & \em{\textbf{NA}}\\
\hline
\em{\textbf{DWR}} & \em{\textbf{My group uses a customized application for Windows (MOPED) to collect and save water quality data during our field runs as well as data from our bbe FluoroProbe. We also use an iPad to record field data on a PDF when this is not available.}} & \em{\textbf{Windows Desktop computer, Apple iPad.}} & \em{\textbf{MOPED (custom software for DWR), Adobe Acrobat for iPad, FluoroProbe custom softwater (bbe moldaenke).}} & \em{\textbf{Convenience (iPad) and robust data applications (MOPED).}}\\
\hline
ICF & Wetlands, nesting bird surveys, Aquatic species surveys, botanical surveys, wildlife surveys, arborist surveys, carcass surveys, other custom data collection efforts.  All are able to collect point, line and polygon data. & All iPad models.  Found that 64gig models are sufficient.  Recommend purchasing cellular models in order to get built-in GPS.  Don't need to activate cellular network to use GPS. & iFormbuilder, Survey123, Fieldmaps, Collector, Fulcrum, Excel, Adobe PDF, Zoho & Software reasons include, free with Esri licensing, customization, robust capabilities for automating reporting, robust mapping capabilities, survey grade mapping.

\em{\textbf{Hardware - Apple iPad/iPhones.  They tend to be more stable and support a larger more robust suite of app capabilities.  Easier to manage devices of the same make/model then a variety of devices.}} & \em{\textbf{NA}} & \em{\textbf{NA}} & \em{\textbf{NA}} & \em{\textbf{NA}}\\
\hline
\em{\textbf{CDFW}} & \em{\textbf{ArcGIS QuickCapture - Application collects GPS coordinates during our aerial survey flights as well as tracking out flight path.}} & \em{\textbf{Samsung Galaxy Tab S3, iPad 9th gen, Iphone - all supported models}} & \em{\textbf{ArcGIS QuickCapture has an online editor to edit data collection application for smartphones or tablets.}} & \em{\textbf{ArcGIS QuickCapture works on any current smartphone or tablet so the list of devices used with this application is because our program had it available.}}\\
\hline
CDFW & Tablet devices were used to collected commercial fishery landings data as well as well as basic data from collecting biological data from those landings. & ASUS Transformer Book T100HA-C4-GR 10. 1 - inch 2 in 1 touchscreen laptop (Cherry Trail Quad-Vore Z8500 Processor, 4GB RAM, 64 GB Storage) & Windows 10 and Microsoft Access & We did look at Dell Venue 10 Pro and iPad Air 2, but most the ASUS had the most RAM for the price which made it faster than all other options under \$500 and it had the best battery life I’ve found, even of more expensive models (10-12 hours), 

\em{\textbf{Our data was stored in Access so we needed something that would run Access data entry forms for easier upload into our larger database systems.}} & \em{\textbf{NA}} & \em{\textbf{NA}} & \em{\textbf{NA}} & \em{\textbf{NA}}\\
\hline
\em{\textbf{NEON}} & \em{\textbf{e-devices are used to collect field and lab data and metadata for about 100 of over 180 data products produced by NEON (freely available on data.neonscience.org). All of our observational data products use e-devices for at least part of the data collection and the aquatic instrument field calibration and maintenance data is collected on e-devices. I am happy to go into more details if there are questions about specific types of data that we collect. There is a lot of variety!}} & \em{\textbf{We use a variety of devices supported by our IT department that run iOS, android, windows, and apple OS. These include ipads (large and mini), iphones, android tablets (these are being phased out possibly), PC, and mac laptops.}} & \em{\textbf{We primarily use Fulcrum (www.fulcrumapp.com) for field data collection. However, we also use other software for interacting and troubleshooting sensors in the field, such as lab view and putty. There might be some others if I really dug into it, but they are more specific- than general-purpose. One of the biggest benefits that we have found with Fulcrum is the ability to write custom javascript code for validation of data prior to ingest into our database and create widgets and warnings for field scientists to address while collecting data. I think there are a lot of fee options that can be used like fulcrum to build forms, but don't have a lot of familiarity with them since we don't use them. Our data is ingested into our own database from the fulcrum cloud database on a nightly basis with different delays depending on the field data collection procedures. Also happy to demo or answer other questions!}} & \em{\textbf{Hardware decisions are made by the IT and Field Science departments for NEON. I have no direct involvement in that choice. I could reach out to folks for more details if that would be helpful.}}\\
\hline
\end{tabular}
\end{table}

\hypertarget{survey123}{%
\section{Survey123}\label{survey123}}

\begin{table}
\centering
\begin{tabular}[t]{>{}l||>{\raggedright\arraybackslash}p{20em}}
\hline
Items & Features\\
\hline
\textbf{Item 1} & \cellcolor{yellow}{Lorem ipsum dolor sit amet, consectetur adipiscing elit. Proin vehicula tempor ex. Morbi malesuada sagittis turpis, at venenatis nisl luctus a.}\\
\hline
\textbf{Item 2} & \cellcolor{yellow}{In eu urna at magna luctus rhoncus quis in nisl. Fusce in velit varius, posuere risus et, cursus augue. Duis eleifend aliquam ante, a aliquet ex tincidunt in.}\\
\hline
\textbf{Item 3} & \cellcolor{yellow}{Vivamus venenatis egestas eros ut tempus. Vivamus id est nisi. Aliquam molestie erat et sollicitudin venenatis. In ac lacus at velit scelerisque mattis.}\\
\hline
\end{tabular}
\end{table}

  \bibliography{book.bib,packages.bib}

\end{document}
