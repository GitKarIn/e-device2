% Options for packages loaded elsewhere
\PassOptionsToPackage{unicode}{hyperref}
\PassOptionsToPackage{hyphens}{url}
%
\documentclass[
]{book}
\usepackage{amsmath,amssymb}
\usepackage{iftex}
\ifPDFTeX
  \usepackage[T1]{fontenc}
  \usepackage[utf8]{inputenc}
  \usepackage{textcomp} % provide euro and other symbols
\else % if luatex or xetex
  \usepackage{unicode-math} % this also loads fontspec
  \defaultfontfeatures{Scale=MatchLowercase}
  \defaultfontfeatures[\rmfamily]{Ligatures=TeX,Scale=1}
\fi
\usepackage{lmodern}
\ifPDFTeX\else
  % xetex/luatex font selection
\fi
% Use upquote if available, for straight quotes in verbatim environments
\IfFileExists{upquote.sty}{\usepackage{upquote}}{}
\IfFileExists{microtype.sty}{% use microtype if available
  \usepackage[]{microtype}
  \UseMicrotypeSet[protrusion]{basicmath} % disable protrusion for tt fonts
}{}
\makeatletter
\@ifundefined{KOMAClassName}{% if non-KOMA class
  \IfFileExists{parskip.sty}{%
    \usepackage{parskip}
  }{% else
    \setlength{\parindent}{0pt}
    \setlength{\parskip}{6pt plus 2pt minus 1pt}}
}{% if KOMA class
  \KOMAoptions{parskip=half}}
\makeatother
\usepackage{xcolor}
\usepackage{color}
\usepackage{fancyvrb}
\newcommand{\VerbBar}{|}
\newcommand{\VERB}{\Verb[commandchars=\\\{\}]}
\DefineVerbatimEnvironment{Highlighting}{Verbatim}{commandchars=\\\{\}}
% Add ',fontsize=\small' for more characters per line
\usepackage{framed}
\definecolor{shadecolor}{RGB}{248,248,248}
\newenvironment{Shaded}{\begin{snugshade}}{\end{snugshade}}
\newcommand{\AlertTok}[1]{\textcolor[rgb]{0.94,0.16,0.16}{#1}}
\newcommand{\AnnotationTok}[1]{\textcolor[rgb]{0.56,0.35,0.01}{\textbf{\textit{#1}}}}
\newcommand{\AttributeTok}[1]{\textcolor[rgb]{0.13,0.29,0.53}{#1}}
\newcommand{\BaseNTok}[1]{\textcolor[rgb]{0.00,0.00,0.81}{#1}}
\newcommand{\BuiltInTok}[1]{#1}
\newcommand{\CharTok}[1]{\textcolor[rgb]{0.31,0.60,0.02}{#1}}
\newcommand{\CommentTok}[1]{\textcolor[rgb]{0.56,0.35,0.01}{\textit{#1}}}
\newcommand{\CommentVarTok}[1]{\textcolor[rgb]{0.56,0.35,0.01}{\textbf{\textit{#1}}}}
\newcommand{\ConstantTok}[1]{\textcolor[rgb]{0.56,0.35,0.01}{#1}}
\newcommand{\ControlFlowTok}[1]{\textcolor[rgb]{0.13,0.29,0.53}{\textbf{#1}}}
\newcommand{\DataTypeTok}[1]{\textcolor[rgb]{0.13,0.29,0.53}{#1}}
\newcommand{\DecValTok}[1]{\textcolor[rgb]{0.00,0.00,0.81}{#1}}
\newcommand{\DocumentationTok}[1]{\textcolor[rgb]{0.56,0.35,0.01}{\textbf{\textit{#1}}}}
\newcommand{\ErrorTok}[1]{\textcolor[rgb]{0.64,0.00,0.00}{\textbf{#1}}}
\newcommand{\ExtensionTok}[1]{#1}
\newcommand{\FloatTok}[1]{\textcolor[rgb]{0.00,0.00,0.81}{#1}}
\newcommand{\FunctionTok}[1]{\textcolor[rgb]{0.13,0.29,0.53}{\textbf{#1}}}
\newcommand{\ImportTok}[1]{#1}
\newcommand{\InformationTok}[1]{\textcolor[rgb]{0.56,0.35,0.01}{\textbf{\textit{#1}}}}
\newcommand{\KeywordTok}[1]{\textcolor[rgb]{0.13,0.29,0.53}{\textbf{#1}}}
\newcommand{\NormalTok}[1]{#1}
\newcommand{\OperatorTok}[1]{\textcolor[rgb]{0.81,0.36,0.00}{\textbf{#1}}}
\newcommand{\OtherTok}[1]{\textcolor[rgb]{0.56,0.35,0.01}{#1}}
\newcommand{\PreprocessorTok}[1]{\textcolor[rgb]{0.56,0.35,0.01}{\textit{#1}}}
\newcommand{\RegionMarkerTok}[1]{#1}
\newcommand{\SpecialCharTok}[1]{\textcolor[rgb]{0.81,0.36,0.00}{\textbf{#1}}}
\newcommand{\SpecialStringTok}[1]{\textcolor[rgb]{0.31,0.60,0.02}{#1}}
\newcommand{\StringTok}[1]{\textcolor[rgb]{0.31,0.60,0.02}{#1}}
\newcommand{\VariableTok}[1]{\textcolor[rgb]{0.00,0.00,0.00}{#1}}
\newcommand{\VerbatimStringTok}[1]{\textcolor[rgb]{0.31,0.60,0.02}{#1}}
\newcommand{\WarningTok}[1]{\textcolor[rgb]{0.56,0.35,0.01}{\textbf{\textit{#1}}}}
\usepackage{longtable,booktabs,array}
\usepackage{calc} % for calculating minipage widths
% Correct order of tables after \paragraph or \subparagraph
\usepackage{etoolbox}
\makeatletter
\patchcmd\longtable{\par}{\if@noskipsec\mbox{}\fi\par}{}{}
\makeatother
% Allow footnotes in longtable head/foot
\IfFileExists{footnotehyper.sty}{\usepackage{footnotehyper}}{\usepackage{footnote}}
\makesavenoteenv{longtable}
\usepackage{graphicx}
\makeatletter
\def\maxwidth{\ifdim\Gin@nat@width>\linewidth\linewidth\else\Gin@nat@width\fi}
\def\maxheight{\ifdim\Gin@nat@height>\textheight\textheight\else\Gin@nat@height\fi}
\makeatother
% Scale images if necessary, so that they will not overflow the page
% margins by default, and it is still possible to overwrite the defaults
% using explicit options in \includegraphics[width, height, ...]{}
\setkeys{Gin}{width=\maxwidth,height=\maxheight,keepaspectratio}
% Set default figure placement to htbp
\makeatletter
\def\fps@figure{htbp}
\makeatother
\setlength{\emergencystretch}{3em} % prevent overfull lines
\providecommand{\tightlist}{%
  \setlength{\itemsep}{0pt}\setlength{\parskip}{0pt}}
\setcounter{secnumdepth}{5}
\ifLuaTeX
  \usepackage{selnolig}  % disable illegal ligatures
\fi
\usepackage[]{natbib}
\bibliographystyle{apalike}
\IfFileExists{bookmark.sty}{\usepackage{bookmark}}{\usepackage{hyperref}}
\IfFileExists{xurl.sty}{\usepackage{xurl}}{} % add URL line breaks if available
\urlstyle{same}
\hypersetup{
  pdftitle={Electronic Data Entry Options for IEP Surveys},
  pdfauthor={IEP DUWG `e-Device' Sub-group led by Karrin Alstad},
  hidelinks,
  pdfcreator={LaTeX via pandoc}}

\title{Electronic Data Entry Options for IEP Surveys}
\author{IEP DUWG `e-Device' Sub-group led by Karrin Alstad}
\date{2023-06-08}

\begin{document}
\maketitle

{
\setcounter{tocdepth}{1}
\tableofcontents
}
\hypertarget{preamble}{%
\chapter*{Preamble}\label{preamble}}
\addcontentsline{toc}{chapter}{Preamble}

This bookdown document is intended as a repository of technical information related to the use of electronic field data entry tools, including reviews of different software options, hardware, and factors regarding interfacing with external sensor inputs and downstream databases.

The initial information was gleaned from interviews and presentations organized during a DUWG e-device focus group (6/2022-6/2023) but these reports are far from exhaustive of the many software/hardware options that are available for field data entry. Meanwhile communication and cloud processing technology is advancing rapidly.

This initiative generally asks that all experience IEP survey teams, or those that are adding digital data-entry tools to their program, please share these experiences and help to develop this resource for other IEP surveys seeking to update data entry protocols,

\hypertarget{bookdown-file-organization}{%
\section{Bookdown File Organization}\label{bookdown-file-organization}}

This document is configured using the bookdown::bs4\_book format for HTML output and the file organization is as follows:

\begin{itemize}
\item
  The main project folder (``edevice2/'') contains all files included in this bookdown document, including the data tables and scripts used to populate the document tables. The main folder of the project contains all of the .Rmd files that compose one (and only one) chapter.
\item
  01-intro.Rmd, 02-survey.Rmd, etc.: These are the chapter files of the e-device book, which are numbered in the order that they appear in the table of contents. A chapter \emph{must} start with a first-level heading: \texttt{\#\ A\ good\ chapter}, and can contain one (and only one) first-level heading.
\item
  \_bookdown.yml: This file contains the configuration options f the e-device book, including the output format, the location of the chapters, and the order of the chapters {[}if numbers are not used{]}.
\item
  \_output.yml: This file contains the configuration options for the output format, such as the theme, the CSS, and the JavaScript.
\item
  index.Rmd: This file contains the content for this current page providing this orientation material. It also contains the front matter for the book, such as the title, the author, and the date, as well as the table of contents.
\item
  \_book/: This is the directory where the compiled book is stored. It contains all of the HTML, CSS, and JavaScript files that make up the book. Files in the \_book directory should NOT be edited.
\item
  images/: This is the directory that contains all static images (images not derived by R code chunk) that are included in the e-device book. The static images (e.g., screen clips) are organized in folders by chapter.
\item
  data\_scripts/: This is the directory that contains all data handing scripts used in the e-device book. Currently, one data read script is run prior to rendering the book. This script output .rds data tables to the main edevice2 folder for access by the Rmd files.
\item
  tables/: Excel files associated with the tables generated in this bookdown document are located in the tables/ folder.
\item
  style.css: contains css formatting instructions. Beyond default settings, a couple of css codes for table formatting have been added.
\end{itemize}

\hypertarget{how-to-contribute-two-ways}{%
\section{How to Contribute: Two Ways}\label{how-to-contribute-two-ways}}

\hypertarget{submit-changes-to-the-e-device-package-using-the-github-interface}{%
\subsection{Submit changes to the e-device package using the Github interface:}\label{submit-changes-to-the-e-device-package-using-the-github-interface}}

\begin{enumerate}
\def\labelenumi{\arabic{enumi}.}
\item
  Create a fork off of the primary package: \url{https://github.com/GitKarIn/e-device2.git}
\item
  Clone this fork to your own system
\item
  Render the book:
\end{enumerate}

\begin{itemize}
\item
  Click on the \textbf{Build} pane in the RStudio IDE and Click on \textbf{Build Book} tool
\item
  Or, build the book from the R console:
\end{itemize}

\begin{Shaded}
\begin{Highlighting}[]
\NormalTok{bookdown}\SpecialCharTok{::}\FunctionTok{render\_book}\NormalTok{()}
\end{Highlighting}
\end{Shaded}

\begin{enumerate}
\def\labelenumi{\arabic{enumi}.}
\setcounter{enumi}{3}
\item
  Compose changes or updates to the book contents
\item
  Check for merge conflicts and submit a pull request
\end{enumerate}

\hypertarget{send-proposed-changes-or-additions-directly-to}{%
\subsection{Send proposed changes or additions directly to:}\label{send-proposed-changes-or-additions-directly-to}}

\href{mailto:karrin.alstad@wildlife.ca.gov}{\nolinkurl{karrin.alstad@wildlife.ca.gov}}

\hypertarget{summary-for-managers}{%
\chapter*{Summary for Managers}\label{summary-for-managers}}
\addcontentsline{toc}{chapter}{Summary for Managers}

\hypertarget{summary-findings}{%
\section{Summary Findings}\label{summary-findings}}

\hypertarget{what-still-needs-to-be-reviewed}{%
\section{What Still Needs To Be Reviewed}\label{what-still-needs-to-be-reviewed}}

  \bibliography{book.bib,packages.bib}

\end{document}
